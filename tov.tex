\documentclass{article}
\usepackage{lmodern}
\usepackage{amssymb,amsmath}
\usepackage[utf8]{inputenc}
\usepackage[T1]{fontenc}
\usepackage{textcomp}
\usepackage{hyperref}
\usepackage{xcolor}
\usepackage{parskip}
\usepackage{csquotes} % or \usepackage{quote}
\usepackage{geometry}
\geometry{margin=1in}
\setlength{\parindent}{0pt} % No indentation
\setlength{\parskip}{1em} % Space between paragraphs

\title{\textbf{Transfer of Value (TOV) Tokens: From Meme to Money}}
\author{By Coywolf}
\date{}

\begin{document}

\maketitle

\begin{quote}
\textbf{ABSTRACT}: In this litepaper, we a) formally establish the emergent blockchain concept of a Transfer of Value token (``TOV'') b) spotlight attributes of prominent tokens that prevent them from achieving the functionality of ``pure'' TOVs c) argue that fair launch memecoin/community tokens on high-throughput Proof of Stake chains provide the best ``ingredients'' for organically evolving into a TOV.
\end{quote}

\section*{INTRODUCTION}

We are living through a financial revolution, reorganization, and reckoning. In only 15 years, Blockchain has radically restructured global value paradigms. Bitcoin shattered state monopolization of currency issuance. DeFi shattered Wall Street monopolization of asset management. USDC and Tether shattered the banks' monopolization of bestowing access to the US dollar, the world's most powerful currency.

By launching the epoch of value decentralization, blockchain has rewritten the rules as to who can create value, who can connect to that value, and who can build those connections.

Yet the revolution remains young and incomplete. Blockchain's rule rewriting has been drafted in pencil, yet to be permanently inked in pen. Viewed within the scale of global finance and commerce, Bitcoin, DeFi, and stablecoins remain robust experiments exhibiting promising results, but experiments all the same.

To achieve blockchain's fullest potential - building an uncancellable global framework for trusted value connections - requires the dawning of the age of Transfer of Value (TOV) tokens.

\section*{WHAT IS A TRANSFER OF VALUE TOKEN?}

In the broadest possible definition, a Transfer of Value token is simply a token one party (a sender) can transmit to another party (a recipient) for the purposes of value exchange.

Optimally, a Transfer of Value token possesses the following attributes:

\begin{itemize}
\item The transfer is fast: To transfer value effectively, at global scale, a TOV's sending, receiving, and on-chain confirmation must take place within less than a second, even if tens of thousands of other parties are attempting similar transactions simultaneously.
\item The transfer is low-cost: To transfer value effectively, at global scale, a TOV's sending, receiving, and on-chain confirmation must cost a negligible fraction of the value transfer itself.
\item The token supply is hard-capped: A token with fluctuating fundamentals renders it impossible for the market of global retail users to collectively and confidently price token valuation via their acts of exchange and usage. Hence, it is mandatory for a TOV to have a hard-cap with no remint function in order to place the market, not engineers, VCs, or quants, in control of token valuation (and thus revaluation).
\item The token supply is widely, fairly, and fully distributed: Tokens whose issuance or mining rewards are predominantly dominated by few parties are priced based on artificial scarcity rather than true value. It comes as no surprise that a Transfer of Value token's price must reflect how well it Transfers Value. If whales, miners, or Wall Street are the predominant owners of a token then the token is being used by its primary holders for Store of Value and speculation (and frequently market manipulation) more than Transfer of Value.
\item The token contract is not alterable by any individual nor centralized party: Obviously ``rugs'' in which code is altered for the purposes of theft or fraud are unacceptable. Nearly as unacceptable is overwriting the implied social contract between token holder and project, exactly what occurs when a developer alters a token's smart contract. The implied agreement of a TOV between all holders is that its price will be determined by the collective decentralized efforts of the TOV's holders rather than the centralized efforts of developers.
\item The token's primary purpose is for Transfer of Value: While many tokens (including NFTs) have been ``fuzzily'' used for Transfer of Value, a true TOV exists for the primary purpose of transferring value between two or more parties who agree the TOV token first and foremost represents a digital instrument of value transfer.
\item The token is future-friendly: With AI, climate crisis, geopolitical upheaval, and a host of other challenges barrelling down upon the global community, a TOV token at minimum must not exacerbate current issues either directly or through obvious second-order effects. Otherwise, it runs the risk of drastic devaluation as its digitally financialized reality butts up against the universe's hard-coded material reality.
\end{itemize}

In short, a Transfer of Value token (TOV) is an on-chain token that an uncoerced and uncoordinated collective has determined through repeated organic use to primarily exist as a medium of value exchange. While many prominent blockchain/crypto projects and their supporters would claim they function as TOVs, to date no true Transfer of Value token exists.

\section*{WHY PROMINENT CRYPTO TOKENS ARE NOT TOVs}

While many tokens have been fuzzily used as Transfer of Value tokens by early blockchain adopters, including Bitcoin, Ethereum, stablecoins, and even NFTs, none have achieved pure TOV status due to deficiencies in their core attributes.

For those who dispute this fact, understand that if a TOV currently existed traditional finance would be in utter free-fall if not already dispatched to the dustbin of history. When a true TOV emerges it will signify the arrival of a world run by the new empire of the blockchainified (and likely AI-ified) global Internet and its collectives, not the national superpowers of the US or China\footnote{Hence the fierce regulatory pushback from all superpowers. A state has a monopoly on violence and the right to issue currency. Lose the former and you lose order. Lose the latter and you lose buy-in from your citizenry to accept the rules backstopped by the former. Suppressing the rise of decentralized finance is an existential necessity for national superpowers to maintain their apex positioning without radical restructuring of their institutions.} nor the corporate superpowers of Apple, Google, Facebook, Amazon, or Tencent.

The following is a brief analysis detailing why existing tokens have failed to achieve TOV status. This analysis consists of restating the optimal attributes of a TOV then placing the token under the microscope to examine its shortcomings as a pure TOV.

\textbf{Bitcoin (\$BTC)}

\begin{itemize}
\item \textbf{The transfer is fast:} False. Bitcoin does not transact at speeds remotely capable of purchasing a cup of coffee. In an instance where 40,000 people want to purchase coffee simultaneously the cups would chill cold for the overwhelming majority of buyers before their Bitcoin transaction executed.
\item \textbf{The transfer is low-cost:} False. The cost of paying for a cup of coffee with \$BTC could range between \$4 and \$60 dollars, depending on chain usage.
\item \textbf{The token supply is hard capped:} Debatable, but for the sake of discussion we can assume that no more than 21 million BTC will ever be minted.
\item \textbf{The token supply is widely, fairly, and fully distributed:} False. While approximately 46 million wallets hold at least \$1 in bitcoin, \$BTC ownership is overwhelmingly clustered around miners, whales, and soon, ETFs. Furthermore, there are nearly two million more \$BTC to be mined, nearly all of which will end up in the wallets of multimillion dollar mining corporations, not rank-and-file retail users.
\item \textbf{The token contract is not alterable by any individual nor centralized party:} Again, debatable, but Bitcoin governance makes significant alterations to the code highly unlikely.
\item \textbf{The token's primary purpose is for Transfer of Value:} Satoshi's intent for Bitcoin to become the global Peer-To-Peer Electronic Cash System, aka a true TOV, was not achieved. Instead, it has become more of a ``digital gold'', a hybrid of a Store of Value and a highly speculative asset.
\item \textbf{The token is future-friendly:} False. In a gross but accurate oversimplification, Bitcoin's Proof of Work consensus mechanism transforms electricity into Bitcoins. This creates an economics incentivizing vast energy expenditure to a) mine a single \$BTC and b) ensure secure network operations. In a world of AI acceleration, electric car proliferation, and ascending nations' citizenry seeking to improve their quality of life, energy is the bottom line scarce asset, not Bitcoin. By its design, Bitcoin is locked in a gladiatorial battle for the future of wattage with godlike superintelligence, mankind's transportation needs, and billions of global citizens.
\end{itemize}

\textbf{Stablecoins}\footnote{For this analysis ``stablecoins'' refers to asset-backed stablecoins, rather than algorithmic stablecoins. Algorithmic stablecoins have yet to be achieved in any significant form, have led to catastrophic failures, and the most successful, Maker's \$DAI, is predominantly backed by asset-backed stablecoins.} \textbf{(\$USDC)\footnote{For the purposes of this analysis, we chose Circle's \$USDC over Tether's \$UDST due to its institutional backing, published auditing, and all around professional superiority. While many consider Tether's cloaked operations to be a feature rather than a bug, the odds of \$USDT being a centrally manipulated token are nonzero as it has already previously admitted to instances of being unbacked and manipulated, and thus unworthy of serious consideration by dedicated futurists unblinded by bag bias.}}

\begin{itemize}
\item \textbf{The transfer is fast:} True. While it is still too slow on Ethereum, next-gen L1s such as Solana and Avalanche and L2s such as Base can transfer \$USDC at thousands of transactions per second. Furthermore, this speed is set to skyrocket as advancements are pushed.
\item \textbf{The transfer is low-cost:} True. USDC transfers are normally under \$.10 per, with the costs significantly decreasing on L2s or with account abstraction architectures that subsidize the gas cost of a transfer.
\item \textbf{The token supply is hard capped:} False. \$USDC is issued by Circle and its pegged asset, USD, is issued by the private/public alliance of US Banks and the US Government.
\item \textbf{The token supply is widely, fairly, and fully distributed:} True. Anyone with an internet connection who wants \$USDC can purchase tokens. Due to \$USDC being pegged to the US dollar, the purchase or sale of \$USDC by any buyer or seller does not impact the price of other \$USDC tokens.
\item \textbf{The token contract is not alterable by any individual nor centralized party:} False. \$USDC is operated by Circle. While Circle is by all appearances a well-run organization, they can freeze assets on their own volition or be ordered to by entities exerting influence over them, such as the US government, Circle's financial backers, or Circle's logistical partners. As such, USDC is ``cancellable'' by outside parties beyond the scope of users' intended TOV exchange.
\item \textbf{The token's primary purpose is for Transfer of Value:} True. The US dollar's primary purpose is as a currency used for the Transfer of Value. As a tokenized version of the US Dollar, \$USDC's primary purpose is also for the Transfer of Value.
\item \textbf{The token is future-friendly:} Mostly true. \$USDC runs on blockchains that operate on energy-efficient POS consensus mechanisms. \$USDC's two primary future risks are a) the US government issuing a Central Bank Digital Currency and legislating all other US-pegged digital tokens illegal b) the devaluation of the US dollar, resulting in the token being pegged to a depreciating asset. Both outcomes are unlikely in the immediate term, but increase with time.
\end{itemize}

\textbf{SOLANA (\$SOL)\footnote{Solana was chosen as an example of a Proof of Stake token rather than a) Ethereum because it is significantly faster and more affordable than Ethereum b) Base or other L2s because they currently operate under centralized control and thus are not true blockchains c) other Proof of Stake L1s because Solana is currently the most valuable non-Ethereum PoS Token.}}

\begin{itemize}
\item \textbf{The transfer is fast:} True. As of this writing, Solana is blazingly fast (when transactions execute and the chain is functional).
\item \textbf{The transfer is low-cost:} True. A basic transfer of \$SOL on Solana would cost a fraction of a penny.
\item \textbf{The token supply is hard capped:} False. \$SOL instituted an inflation rate of 8\% to decrease by 15\% per year until reaching a steady state of 1.5\%. This will lead to price fluctuations that will be gamed by expert operators at the expense of retail users. This leaves pricing heavily influenced by future issuance rather than usage.
\item \textbf{The token supply is widely, fairly, and fully distributed:} Debatable. There are certainly Solana whales (as well as VC and Solana Foundation dominance) but due to being Proof of Stake rather than Proof of Work its tokenomics do not suffer from the malign influence of centralized mining outfits as is the case with Bitcoin.
\item \textbf{The token contract is not alterable by any individual nor centralized party:} False. Solana Labs could effectively shut down Solana tomorrow if they saw fit. While Solana may continue to technically operate due to its current validators and ecosystem builders rising to the challenge, Solana's progress and \$SOL's value would suffer a sharp and likely fatal reversal of fortune.
\item \textbf{The token's primary purpose is for Transfer of Value:} False. \$SOL's primary purpose is the utility of running the Solana blockchain, through staking and through gas. This means that every \$SOL staked or used to pay gas is a \$SOL that can't be used as a TOV.
\item \textbf{The token is future-friendly:} Largely true. There's no need to debate the virtues of monolithic vs modular chains in a litepaper on Transfer of Value tokens. It suffices to say that Solana is far more future friendly than Bitcoin and has a dedicated company and ecosystem of builders working to keep it operating at the vanguard as opposed to a mysteriously pseudonymous founder who disappeared into a fog of history (Satoshi).
\end{itemize}

To conclude this section, neither Bitcoin, nor stablecoins, nor Proof of Stake L1 tokens satisfy the requirements necessary for a token to achieve true TOV status nor could they grow past these deficiencies as they are baked-in to their very nature.

\section*{WHY A MEMECOIN COULD BECOME THE ONE TOV TO RULE THEM ALL}

First, what is a memecoin (or ``community coin'')? For this litepaper, we will define a memecoin as a token launched with no intended utility beyond accruing value due to how popularly its memetic attributes spread through global culture to stimulate purchasing demand.

Some snarky critics from the crypto Twitterati claim that all tokens are memecoins. A more reasoned stance would be to posit that a significant portion of the value of any token is derived from its memetic attributes. This is hard to dispute, as:

\begin{itemize}
\item No publicly traded blockchain projects beside stablecoins have achieved the Product Market Fit of frequent real world use
\item Precious few blockchain projects (besides stablecoins) consistently achieve profitable returns due to their business construction
\item Crypto writ large has achieved over a two trillion dollar market capitalization
\end{itemize}

It is a surreal state of affairs to say the least, as if the rules of traditional market based economics have been tossed out the window into zero financial gravity. This is because crypto was an industry born in the era of ``money printer go BRRR'' Zero Interest Rate Phenomenons, where its price-determining financial ``fundamentals'' index more heavily on attention than profits.

Thus the brand popularity of a blockchain project becomes its primary value rather than what it can actually do. This popularity is won by winning mindshare via meme wars that pit the ``conceptual identity DNA'' of one project against another in the digitally Darwianian natural law of crypto's PvP arena.

How do memetic themes manifest in blockchain? ``One bitcoin = One bitcoin'' Laser eyes. Pepe the frog. ``Have Fun Staying Poor.'' ``WAGMI.'' ``GM.'' ``Modular composability.'' ``Monolithic architecture.'' ``Subnets.'' While some memes lean more cartoonishly untethered to reality and others more compsci PHD moonshot, each of these snippets of ``conceptual identity DNA'' will be immediately recognizable to any crypto native.

With apologies to the august engineers and biz dev teams at legitimate blockchain labs, these memes fuel the speculation that drives price action across the current industry far more than any Github push or partnership.\footnote{As is standard with bubbles. Once the blockchain industry begins to achieve real world product market fit then price action will ruthlessly eliminate projects that don't by reallocating their resources to those that do. Fortunately, many promising projects hint at achieving this within the next 3 years.}

With the importance of memes to blockchain firmly established, now we can begin to assert our most audacious claim: that if a true TOV will be achieved and become a digital global money, it will originate as a memecoin.

Before we revisit the analysis applied to the prominent tokens within the context of memecoins we must first briefly overview the current state of memecoins.

\textbf{There Are Thousands Upon Thousands of Memecoins}

Imagine the quest to become the world's Transfer of Value Token as a horse race between every token currently in existence. In this hypothetical, the field of horses would predominantly consist of memecoins. While some may say that Bitcoin is Secretariat and has already run away with the Kentucky Derby with an insurmountable lead, we have already proven this to be a bag-biased fatuous argument. Few if any use Bitcoin as a TOV and its underlying tech stack does not support TOV usage.\footnote{For Bitcoin Maxis who assert that Lightning scaling solutions will solve this we simply remind them that the greatest blockchain engineering minds of this generation have dedicated their lives to working on projects besides Lightning, despite BTC's \$1.2 trillion dollar market cap at the time of this writing. For example, when Patrick O'Grady departed Coinbase, America's largest Bitcoin custodian by orders of magnitude, it was to run engineering at Avalanche rather than to scale Bitcoin via Lightning.} Thus it would be wise to ``bet the field.''

\textbf{Minting Them Is Incredibly Simple}

During crypto's mini-bull in the first half of 2024 Solana transformed into a memecoin casino. This was in no small part due to the fact that it became tremendously easy for anybody with a modicum of developer skills to copy existing memecoin code, change the name, and launch their own token. Even if the future ``one TOV to rule them all'' has yet to be minted into existence the barrier to memecoin entry is practically nonexistent for the innovator who understands cultural economics as well as Satoshi understood cryptography.

\textbf{Memecoins Achieve Significant Buy-In Fast}

As of this writing, the two most popular ur-memecoins, Dogecoin and Shiba Inu, have a combined market cap approaching that of Uber. Within less than a year and a half, the \$BONK token on Solana surged from a goofy nothingburger to a \$2.5 billion market cap. While the kludgy state of current blockchain UX still limits mainstream retail adoption, dramatic improvements such as account abstraction and gasless transactions are pickaxing at the dam restraining back a surge of global crypto adoption. Once it bursts, blockchain buy-in will occur at breakneck speed and memecoin adoption can increase at the speed of Web 2.0's social media Cambrian explosion.

\textbf{Genuine Communities Have Begun to Coalesce}

Whichever memecoin ultimately wins the race to become the global TOV token will require fervent digital grassroots community action inspired by populist memetics.

A handful of successful memecoin projects have begun to organically generate their own passionate tribes of supporters who have bought into a) the project's token b) the meme of the project's cultural identity c) each other. While membership in these memecoin tribes are non-exclusive and the tribes frequently exhibit behavioral overlap, they often feature distinctive characteristics that prevent them from being mere cases of cultural copypasta.

For example, within the Avalanche ecosystem's burgeoning memecoin ecosystem the cultural spectrum ranges from COQINU, to KIMBO, to TECH. COQINU is for all intents and purposes a sophomoric phallic joke. KIMBO is an offbeat homage to an Avalanche executive's family dog that boasts surprisingly impressive CGI promotional videos. TECH is a project that launched as a cheeky commentary on memecoins ``tech'' boiling down to ``Number go up technology'' (with regards to price) but has since begun a pivot into positioning to become the community coin of tech-embracing futurists. Each project sprung from comic origins -- and perhaps a dash of get rich quick hopium -- but as their ownership grew they became tokenized manifestations of the distinct relationship dynamics within their communities.

As impediments to global blockchain adoption are removed these most potent of these communities will outreach, innovate, and blitzscale.

Now for the analysis of a memecoin's ability to become a TOV based on the previous criteria:

\textbf{Memecoin (\$TOV)}

\begin{itemize}
\item \textbf{The transfer is fast:} True. Proof of Stake speeds are accelerating to 3x faster than Visa TPS throughput and will continue to decrease as hardware and software breakthroughs accrue. A \$TOV operating on these rails will move as fast if not faster than a digital US Dollar.
\item \textbf{The transfer is low-cost:} True. In fact, they very well could end up free to transfer based on emerging ``tokenomics'' models that would subsidize the cost of transferring.
\item \textbf{The token supply is hard capped:} True. Most memecoins mint once and only once. The rest are scams.
\item \textbf{The token supply is widely, fairly, and fully distributed:} True. Non-scam memecoins practice ``fair launches'' where there are no VC sales nor pre-sales nor team funding prior to a public listing of the token on a decentralized exchange, granting the public the same access as any insiders to purchase tokens.
\item \textbf{The token contract is not alterable by any individual nor centralized party:} True. Non-scam memecoins mint the tokens then burn the contract. This removes the possibility of manipulating the token's value by altering the code.
\item \textbf{The token's primary purpose is for Transfer of Value:} Herein lies the rub. Memecoins' primary purpose is for speculation, the same as nearly all blockchain tokens. However, with significant enough adoption it could cross the chasm into achieving true TOV status without any of the pitfalls of other prominent tokens. The road to mass adoption could take many routes, but here is one hypothetical scenario:
\begin{itemize}
\item \$MEME launches and achieves 8000 holders within two months based on a short-lived market frenzy
\item \$MEME spins up its own token-gated DAO (Decentralized Autonomous Organization). In order to access the DAO you must stake \$MEME.
\item \$MEME becomes a unit of exchange within the DAO for the exchange of goods and services.
\item ``Cash back'' style rewards for these Transfer of Value exchanges are paid in \$MEME so that people can pay in alternate currencies and receive a bonus of \$MEME in return.
\item DAO members from around the world onboard their communities to the DAO, slowly at first, then in droves
\item DAO members do not horde \$MEME because it's more valuable to spend to grow the community than it is to sit on like a ``Store of Value''
\item Various DAOs and blockchain-powered platforms begin to adopt \$MEME as a currency due to its growing adoption and devoted community, interlinking disparate groups around the shared values of \$MEME's ``conceptual identity DNA.''
\item As blockchain becomes the primary value transmission mechanism for the globe \$MEME becomes the primary TOV for blockchain.\footnote{Of course \$MEME serves in the above example is a stand-in for any memecoin currently in existence or yet to be minted}
\end{itemize}
\item \textbf{The token is future-friendly:} Aggressively true. As a fair-launch, no-owner, truly decentralized project, a memecoin answers to nobody but the market. Since it will run on Proof of Stake rather than Proof of Work, the energy requirements to launch and maintain a memecoin would be an insignificant fraction of what is required to operate Bitcoin. Best of all, the meme itself can be a future-friendly meme, inspiring and accelerating global buy-in due to its future-friendliness.
\end{itemize}

\section*{CONCLUSION}

We fully recognize that a litepaper suggesting a memecoin will become the world's Transfer of Value token could easily be dismissed as the ravings of a crypto-deranged mind. Today, the crypto market is largely ``ponzinomic'' and memecoins are amongst the greatest offenders. The blockchain technology does not yet exist to support zipping tokens around the planet at hundreds of thousands of transactions per second. Furthermore, the concept of a Transfer of Value token stands in direct opposition to the intentions of every sovereign nation seeking to control their own economies and influence others through fiat issuance.

Of course, each of these steel men reveal their armor to be filled with straw. Yes, crypto remains a bubble machine of an industry, but once bubbles pop the most antifragile infrastructure and innovators remain.\footnote{The tech that remained after the fiber optic and dot com bubbles burst gave rise to Web 2.0's born of global connectivity that generated unprecedented wealth creation.} That blockchain infrastructure is currently being built by many of this generation's most mission-driven geniuses. And with regards to currency sovereignty, the enemy of my enemy is my friend, and in the brewing economic fiat war of all nations against all nations, the currency of no nation becomes friend to all.

Besides, the true rising superpower is not a nation, but technology itself. As blockchain, AI, quantum computing, nuclear power, space travel, robotics, bioscience, and countless other fields accelerate, the superpower of technology will need its own high-throughput digital currency to transfer value between people, organizations, and nations so they can literally buy into the goal of civilizational progress.

And what better currency than a tokenized meme the world can believe in?

\end{document}
